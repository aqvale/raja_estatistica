\section{Conclusão}

O planejamento foi realizado respeitando os princípios de replicação e aletoriedade, bem como as etapas de execução de um estudo estatístico baseado em \textbf{DoE}.

O estudo estatístico realizado na Seção \ref{sec:resultados} revela a altura como sendo a variável mais relevante na saída do sistema, e o clipe constitui a variável mais expressiva na alteração do momento de inércia do ``helicóptero''. Desse modo, a relação entre os outros parâmetros apresentados não houve nenhuma significância com a coleta das amostras dos testes realizados e o modelo fica adequadamente representado pela Equação \ref{eq2:model1}.

O valor de 78,56\% obtido no parâmetro $R^2$ indica que o modelo consegue explicar por meio de uma relação linear a relação entre as variáveis dependentes e a variável independente de modo satisfatório.

