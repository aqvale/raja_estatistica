\section{Conclusão}

O planejamento foi realizado respeitando os princípios de replicação e aletoriedade, bem como as etapas de execução de um estudo estatístico baseado em \textbf{DoE}.

Embasado pelos resultados obtidos pelo estudo estatístico realizado na Seção \ref{sec:resultados}, pôde-se perceber a altura como sendo a variável mais relevante na saída do sistema, e o clipe, em comparação com os adesivos, constitui a variável mais expressiva na alteração do momento de inércia do ``helicóptero''.

O valor de 78,56\% obtido no parâmetro $R^2$ indica que o modelo consegue explicar por meio de uma relação linear a relação entre as variáveis dependentes e a variável independente de modo satisfatório.